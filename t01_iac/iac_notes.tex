\documentclass[slides]{pgnotes}

\title{Infrastructure as Code (IaC)}

\begin{document}

\maketitle

\section{Infrastructure as Code}\label{infrastructure-as-code}

\begin{itemize}
\item
  Infrastructure as Code in the broad sense refers to stored textual descriptions of system configurations.
\item
  This applies to any computing system (e.g.~desktop configuration, physical server configuration) as well as cloud systems.
\item
  Other terms: desired state configuration.
\end{itemize}
  
\subsection{Imperative vs declarative}\label{imperative-vs-declarative}

IaC can be broadly broken down into imperative / procedural vs declarative.

Consider the basic act of making a cup of tea:
\href{https://stackoverflow.com/questions/1619834/what-is-the-difference-between-declarative-and-procedural-programming-paradigms}{StackOverflow
question})

\begin{description}
\item[Imperative / Procedural]
is where we specify instructions to be carried out sequentially to
achieve the state we want.

\begin{itemize}
\item
  ``(1) Go to kitchen. (2) Get sugar, milk, and tea. (3) Mix them (4)
  heat over the fire till it boils (5) Put that in a cup and bring it to
  me''
\item
  Examples: PowerShell Script, Java program
\end{itemize}
\item[Declarative]
is where we specify the state we want to achieve, rather than how to do
it.

\begin{itemize}
\item
  ``Get me a cup of tea.''
\item
  Examples: SQL, HTML.
\end{itemize}
\end{description}

Many people incorrectly assume IaC to be automatically declarative when
mentioned. Often tools offer a blend of imperative and declarative
input.


\subsection{Requirements}\label{requirements}

For IaC to be an option:

\begin{enumerate}
\item
  Desired state of the system (resources, configuration of each,
  connections to each, permissions) needs to be known in a form that can
  be written down / diagrammed.
\item
  All of the resource providers must allow some method of automated
  configuration E.g. web API (Zoom API), command-line tools (AWS CLI),
  text-based serial console (network hardware).
\end{enumerate}

From now on we will assume that we are trying to automate AWS resource creation only.


\subsection{Tooling}\label{tooling}

\begin{description}
\item[Standard scripting environments]
  including Bash, PowerShell that can utilise the AWS CLI.
\item[General purpose IaC tools]
  like Ansible, CFEngine, Chef, Puppet, cloud-init, EC2 launch that are designed to run on servers (virtual or physical).  
  \begin{itemize}
  \item
    Will meet cloud-init / EC2Launch later on!
  \end{itemize}
\item[Cloud-first IaC tools]
  like AWS CloudFormation or Terraform that are built primarily for automating cloud configurations and can interact directly with cloud providers.
\end{description}

Can encounter more complex configurations that are a hybrid of tools.

\section{Shell-based automation}\label{shell-based-automation}

We have already encountered shell-based automation. Some important practices:

\subsection{Dynamic lookup}\label{dynamic-lookup}

Required names (AMIs, regions, Account ID) should be looked up
dynamically as far as possible. Assume that resources may change. Don't
hard-code anything that can be looked up.

As a side note, it's usually best to look up anything ONCE in your script and re-use that value for the duration of the script.

\subsection{Capture output}\label{capture-output}

AWS returns information in response to commands. Needs to be captured appropriately (using \texttt{ConvertFrom-Json} or \texttt{jq}).

Derived data (like ARNs) are often best looked up rather than created, as the rules may change.

\subsection{Idempotency}\label{idempotency}

An operation is deemed to be idempotent if it can be applied multiple times without changing the result beyond its original application.
In the context of a scripted setup, it should:

\begin{itemize}
\item
  Set up the defined environment if it doesn't already exist.
\item
  Complete the setup of the defined environment if it already partially  exists.
\item
  Do nothing if everything is already as it should be.
\item
  Handles changes if required: either by modification (harder, sometimes
  required), delete / recreation (easier in principle, problems if
  resources contain data (e.g.~S3, databases).
\end{itemize}

Dealing with partial setup and/or changes is difficult. As an
alternative, it should at least prevent re-running script if to do so
would cause clashes.

\subsection{Facilitate troubleshooting}
\label{facilitate-troubleshooting}

Scripts should ease troubleshooting by:

\begin{enumerate}
\item
  \textbf{Labelling resources} as far as possible to facilitate visual
  inspection in GUI / manual CLI. Often done using the Tag facilities.
\item
  \textbf{Logging error messages} when they occur (or at least not
  hiding them!)
\end{enumerate}

\section{CloudFormation}\label{cloudformation}

CloudFormation is a declarative IaC tool available in AWS.
Uses \textbf{templates} to create \textbf{stacks}.

AWS CloudFormation best practices:
https://docs.aws.amazon.com/AWSCloudFormation/latest/UserGuide/best-practices.html


\subsection{Templates}\label{templates}

Templates specifies the resources, their configuration, permissions etc.

\begin{itemize}
\item
  Templates are written in YAML (or JSON).
\item
  Any editor can be used to make a template. As an alternative:

  \begin{itemize}
  \item
    templates can be constructed visually using AWS CloudFormation
    Designer.
  \item
    could write a tool to output the correct template
  \end{itemize}
\end{itemize}

\subsection{Stacks}\label{stacks}

A stack is an instance of a template created by CloudFormation.

\begin{itemize}

\item
  There may be multiple stacks in existence from the same template.
\end{itemize}


\section{CloudFormation example}\label{cloudformation-example}

\subsection{Basic template}\label{basic-template}

Consider we have the file \texttt{s3\_buckets\_template.yaml}:

\inputminted{yaml}{s3_buckets_template.yaml}

\begin{enumerate}
\item
  As a minimum, template will contain a \texttt{Resources} object with
  at least one child.
\item
  Every \texttt{Resource} has a \texttt{Type} from the:
  \href{https://docs.aws.amazon.com/AWSCloudFormation/latest/UserGuide/aws-template-resource-type-ref.html}{AWS
  resource and property types reference}
\item
  Usually contains a \texttt{Properties} object.
  Required properties depend on the resource type.
\end{enumerate}

We can then create a stack based on this template from the console or CLI.

\subsection{Stack creation}\label{stack-creation}

To create a stack based on a given template, specify the template file (on our local computer) and give the stack a name:

\begin{verbatim}
aws cloudformation create-stack --stack-name buckets --template-body file://s3_buckets_template.json
\end{verbatim}

Note:
\begin{enumerate}
\item The \texttt{StackId} is returned as an ARN in the JSON from this command.
\item The command returns immediately but stack creation takes time.
\end{enumerate}

If we list our buckets we can see the newly created buckets from the stack:

\begin{verbatim}
aws s3api list-buckets
\end{verbatim}

Notice how CloudFormation generates a unique name for the buckets derived from the Resource names.

\subsection{List stacks}
\label{sec:list-stacks}

Can list stacks created in our account on the console and command-line:

\begin{verbatim}
aws cloudformation list-stacks
\end{verbatim}


\subsection{Multiple instances}
\label{sec:multiple-instances}

A template can be instantiated multiple times in different stacks:
\begin{itemize}
\item Use same template file (no need to duplicate / copy)
\item Give the stack a different name
\end{itemize}
Things to watch out for:
\begin{enumerate}
\item Avoid forcing resource names (e.g. \texttt{QueueName} property)
\end{enumerate}


\subsection{Stack deletion}\label{stack-deletion}

To delete a stack (i.e.~all of its resources) we can issue the command:

\begin{verbatim}
aws cloudformation delete-stack --stack-name buckets
\end{verbatim}


\section{Drift detection}
\label{drift-detection}

Consider a situation where the actual resources on AWS are modified
outside of CloudFormation. This leads to a situation known as stack
drift. Detecting these is a two-step process

\begin{verbatim}
# detection 
aws cloudformation detect-stack-drift --stack-name buckets

# display
aws cloudformation describe-stack-resource-drifts --stack-name buckets
\end{verbatim}

If, for example we deleted one of the buckets manually:

\begin{verbatim}
# delete bucket (created by cloudformation, but name will vary)
aws s3api delete-bucket --bucket buckets-mypublicbucket-xge67cn5a1kv

# re-do detection and display
aws cloudformation detect-stack-drift --stack-name buckets
aws cloudformation describe-stack-resource-drifts --stack-name buckets

# notice DELETED status on MyPublicBucket
\end{verbatim}

\end{document}
