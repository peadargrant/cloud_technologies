\documentclass[slides]{pgnotes}

\title{Function as a Service}

\begin{document}

\maketitle

\section{Serverless compute}\label{serverless-compute}

\begin{center}
  \includegraphics[width=0.2\linewidth]{lambda_logo}
\end{center}

\textbf{Serverless functions} are the basic PaaS compute component:
\begin{itemize}
\item Sometimes referred to as Function-as-a-Service
\item basic FaaS offering in AWS is called Lambda.
\end{itemize}


\subsection{Lambda components}\label{key-components}

\begin{description}
\item[Function]
  is an AWS resource you can \emph{invoke} to run your own code on AWS infrastructure.
\begin{itemize}
\item
  Lambda runs your custom code on AWS
\item
  No need for you to setup EC2 instances and VPC.
\end{itemize}
\item[Executable]
  consists of \textit{code} within a lambda-provided \textit{runtime}:
  \begin{description}
  \item[Code:]
    provided by you, (source code, bytecode or compiled code).
  \item[Runtime:]
    provided by AWS. (must match code)
\end{description}
\item[Event] \textbf{invokes} the lambda function. Might be:
  \begin{itemize}
  \item explicit \texttt{invoke} at the CLI or console, or
  \item triggered from other AWS services (S3, SNS, API Gateway, SQS polling).
  \end{itemize}
\end{description}

\section{Code}\label{code}

~


\subsection{Runtime}\label{runtime}

Runtime refers to the support AWS Lambda offers for different languages.
Supports offered for different languages varies:

\begin{itemize}
\item
  \textbf{Python and JavaScript are best supported:}
  \begin{itemize}
  \item You can write / edit functions in the AWS Console.
  \item Most examples online involve these languages.
  \end{itemize}
\item
  A number of other languages are fully supported:
  \begin{itemize}
  \item must be authored / edited locally and uploaded in a \emph{deployment package}.
  \item These include: Java, .net (incl. C\# and PowerShell).
  \end{itemize}
\item
  Other languages or legacy code can be run on Lambda by creating a \emph{custom runtime}.
  (Involved process!)
\end{itemize}

\subsection{Deployment package creation}\label{deployment-package-creation}

Code is bundled into a ZIP file, named a \emph{deployment package}. The
precise layout will depend on the chosen runtime.

\begin{itemize}
\item
  You don't need to worry about this when creating code in the AWS Console.
\item
  Interpreted languages (like Python, JS) will need the source files.
\item
  Bytecode-compiled languages (like Java) will consist of the
  bytecode-compiled class files.
\end{itemize}

The relevant files can be ZIPped using:
\begin{itemize}
\item On Windows \texttt{Compress-Archive} cmdlet in PowerShell.
\item On Mac / Linux \texttt{zip} command in Bash/zsh.
\end{itemize}

\section{Execution role}\label{execution-role}

Execution roles are assumed by a lambda function, and grant the function
permissions to use other AWS resources in your account.

Each execution role has a name (e.g. \texttt{helloworld-ex}) from which
we can derive its ARN:

\begin{minted}{powershell}
$ExecutionRoleName="helloworld-ex"
$ExecutionRoleArn="arn:aws:iam::123456789012:role/helloworld-ex"
Write-Host $ExecutionRoleArn
\end{minted}

\subsection{Trust policy}\label{sec:trust-policy}

The trust policy specifies what AWS component(s) can assume the role.

\inputminted{json}{trust_policy.json}

You can use this trust policy as-is for all your lambda code. 

\subsection{Attached policies}\label{attached-policies}

Execution roles can then have policies attached using \texttt{iam}.
Rather than write these from scratch, we will attached AWS Managed
Policies.

\newpage
The simplest policy (\texttt{AWSLambdaBasicExecutionRole}) allows a lambda function to write to CloudWatch Logs:
\inputminted{json}{AWSLambdaBasicExecutionRole.json}
Additional policies can be attached (or created and attached) if the
lambda function needs access to other AWS services.

\section{Hello World example}\label{hello-world-example}

Our simple \texttt{Hello\ World} will just output a message to
CloudWatch logs. It will ignore the input.

The program is a single function, named \texttt{hello\_handler} in a
file named \texttt{hello\_handler.py}. It will run as the execution role
\texttt{hello-ex}.

\subsection{Code}\label{code-1}

We have a python file \texttt{hello\_handler.py} with the following
simple program:

\inputminted{python}{hello_handler.py}

\subsection{Deployment package
creation}\label{deployment-package-creation-1}

To package the code into a ZIP we can:

\begin{minted}{powershell}
Compress-Archive -Path hello_handler.py -DestinationPath hello_code.zip

# issues on Mac/Linux in PowerShell due to file permissions, use instead (in Bash):
zip hello_code.zip hello_handler.py
\end{minted}

\subsection{Execution role creation}\label{execution-role-creation}

Our simple trust policy allows Lambda to assume the role (and will work for most functions), defined in \texttt{trust\_policy.json}:

\inputminted{json}{trust_policy.json}

Then we can create the execution role itself:

\begin{minted}{powershell}
# create the role 
aws iam create-role `
--role-name hello-ex `
--assume-role-policy-document file://trust_policy.json

# attach AWSLambdaBasicExecutionRole (for basic I/O needed by Lambda function)
aws iam attach-role-policy `
--role-name helloworld-ex `
--policy-arn arn:aws:iam::aws:policy/service-role/AWSLambdaBasicExecutionRole
\end{minted}

\subsection{Function creation}\label{function-creation}

The handler parameter specifies the entry point that handles the event.
Its format varies depending on the runtime (language) used. For python
it normally is:

\begin{verbatim}
[file (without extension)].[function name]
\end{verbatim}

\begin{minted}{powershell}
aws lambda create-function `
--function-name hello `
--zip-file fileb://hello_code.zip `
--handler hello_handler.hello_handler `
--runtime python3.8 `
--role arn:aws:iam::123456789012:role/hello-ex 
\end{minted}

\subsection{Invoking the function}\label{invoking-the-function}

We invoke (or run) the function:

\begin{minted}{powershell}
# invoke the function
aws lambda invoke --function-name hello out.txt

# read output produced
Get-Content hello_out.txt 
\end{minted}

\subsection{Updating code}\label{updating-code}

If we want to update the function's code, we can modify the source files
and then:

\begin{minted}{powershell}
# new ZIP file:
Compress-Archive -Force -Path hello_handler.py -DestinationPath hello_code.zip

# update the code on Lambda
aws lambda update-function-code --function-name hello --zip-file fileb://hello_code.zip 
\end{minted}

\subsection{Deletion}\label{deletion}

\begin{minted}{powershell}
aws lambda delete-function --function-name hello
\end{minted}

\section{Input handling}\label{input-handling}

Input is passed in/out of lambda functions using JSON-formatted text.
Python has a built-in dictionary time which the incoming JSON is
transparently converted to. Imagine we modified our hello function to
be:

We expect a firstname and surname input. These are provided as JSON, so
here we make a file for PowerShell to send the lambda function:

Then invoke, this time with the payload given:

\begin{verbatim}
aws lambda invoke --function-name hello `
--payload fileb://payload.json hello_out.txt
\end{verbatim}

\end{document}
