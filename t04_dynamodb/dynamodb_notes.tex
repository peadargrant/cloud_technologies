\documentclass[slides]{pgnotes}

\title{DynamoDB}

\begin{document}

\maketitle

\tableofcontents

\section*{Required reading}\label{sec:required-reading}
\addcontentsline{toc}{section}{Required reading}

\begin{description}
\item[Docs:]
\url{https://docs.aws.amazon.com/amazondynamodb/latest/developerguide/}
\item[Inbuilt help]
for dynamodb:

\begin{itemize}
\item
  \texttt{aws\ dynamodb\ help} to see commnads
\item
  \texttt{aws\ dynamodb\ create-table\ help} for the description,
  synopsis and options
\end{itemize}
\end{description}

\section{Cloud databases}\label{sec:cloud-databases}

The cloud offers a number of persistence options, including both
relational and non-relational databases. The array of choices can be
confusing. Often a number of possible solutions exist for any given
problem. Choosing the most suitable is not straightforward.

We will look at a simple database today, DynamoBD. According to AWS:

\begin{quote}
DynamoDB is a fully managed NoSQL database service that provides fast
and predictable performance with seamless scalability
\end{quote}

\subsection{Use cases}

DynamoDBis a useful PaaS tool for simple data persistence that scales well:
\begin{itemize}
\item It is \emph{NOT} a replacement for relational or other databases.
\item There is much ill-informed comment on the internet touting the merits of different database technologies without any context or objective basis.
\item DynamoDB can be used as a persistence store for:
  \begin{itemize}
  \item applications themselves running with AWS (Lambda, EC2)
  \item applications that run elsewhere (own server, own laptop).
  \end{itemize}
\end{itemize}
  
This class provides a \emph{basic} introduction to DynamoDB.


\section{DynamoDB}\label{sec:dynamodb}

\subsection{Key components}\label{sec:key-components}

A DynamoDB table is a collection of \emph{items}. (e.g.~People) Note
that there is no equivalent of ``database'' grouping multiple tables!

\subsection{Item}\label{sec:item}

An item items is a collection of \emph{attributes}. Each item is a
member of a table. (e.g.~a person) Similar to a row in a relational DB,
CSV file or spreadsheet.

\subsection{Attributes}\label{sec:attributes}

Attributes are the fundamental data element. An attribute maps a key to
a value for that item. (e.g.~name = John) Similar to a column in a
relational DB, CSV file or spreadsheet. Attributes can be:

\begin{description}
\item[Scalar:]
number, string, binary, Boolean, and null.
\item[Set:]
multiple scalar values in a set. Allowed types are string set, number
set, and binary set
\item[Document]
types are list and map, roughly mapping to JSON document types.
\end{description}

Unlike a traditional DB, different items may have a differing set of
attributes. Only the primary key attribute is required, .

\subsection{Primary key}\label{sec:sec:primary-key}

Every item must have a primary key. The primary key is either:

\begin{description}
\item[Simple primary key (Partition key)]
must uniquely identify each item. Name comes from its internal use in a
hash function to distribute table contents among physical storage.

\begin{itemize}
\tightlist
\item
  If data already has a unique ID then it should be the simple primary
  key.
\end{itemize}
\item[Composite primary key (Partition key and sort key):]
must together uniquely identify each item. Partition keys of 2 items can
be same if sort keys differ.

\begin{itemize}
\item
  Partition key determins physical storage location.
\item
  Items with same partition key are stored in ascending order of sort
  key.
\item
  In larger workloads, must ensure partition key is not the same across
  large portions of data set.
\end{itemize}
\end{description}

Only scalar types can be used in primary keys.

\section{Access pattern}\label{sec:access-pattern}

Most server-side databases use a custom binary or text protocol on a
specific port number. DynamoDB is HTTP based and has a web-service API.
This can be used from the AWS CLI or any language that the AWS SDK
supports.

\section{Suitability}\label{sec:suitability}

DynamoDB is a good introduction to cloud-based PaaS database services.
However, its suitability for any given application needs to be
considered carefully:

\begin{itemize}
\item
  Good for single-table applications where language used supports its
  API.
\item
  It is NOT a drop-in replacement for an SQL DB.
\item
  NOT a relational database: Has no foreign keys, no JOINs, no GROUP BY,
  no unique keys.
\item
  Has single-digit millisecond response time.
\item
  Not really designed for ad-hoc queries.
\end{itemize}

\section{Operations}\label{sec:operations}

Assume we want to create a table \texttt{players}. Each item has an
attribute \texttt{name} (string) and will have a second attribute,
\texttt{points} (numeric). The handout assumes that you are looking at
the help text for each command - explanations there will not be repeated
here.

In practice, data manipulation operations (like inserting new data,
deleting data, querying) are likely to come from application code in
Java, C\#, C++ via the AWS SDK rather than via direct CLI / AWS Console.

\subsection{Choice of primary key}\label{sec:choice-of-primary-key}

Here we will have a simple primary key, consisting of the partition key
\texttt{name}. This means that every item in the \texttt{players} table
\emph{must} have a \texttt{name} attribute.

\subsection{Table creation}\label{sec:table-creation}

\begin{verbatim}
$TableName="players"

# create the table
aws dynamodb create-table --table-name $TableName `
--attribute-definitions AttributeName=name,AttributeType=S `
--key-schema AttributeName=name,KeyType=HASH `
--billing-mode PAY_PER_REQUEST
\end{verbatim}

The table creation command is asynchronous:
\begin{itemize}
\item Although it returns, the table may stay in the \texttt{CREATING} status for some time before it becomes \texttt{ACTIVE}.
\end{itemize}

\subsection{Putting item into table}\label{sec:putting-item-into-table}

\begin{verbatim}
# basic usage
aws dynamodb put-item --table-name $TableName
--item '{\"name\": {\"S\": \"John\"}, \"points\": {\"N\": \"10\"}}'
\end{verbatim}

The extra quotation marks are because of how strings need to be encoded
within other strings. See:\\
\url{https://docs.aws.amazon.com/cli/latest/userguide/cli-usage-parameters-quoting-strings.html}

\subsection{Reading single item}\label{sec:reading-single-item}

\begin{minted}{bash}
# retrieve value based on key
aws dynamodb get-item --table-name $TableName --item '{\"name\": {\"S\": \"John\"}}'
\end{minted}

The output is returned as a JSON dictionary representing the item's attributes:

\begin{minted}{powershell}
# can get as PowerShell objects in usual way
$Item = ( aws dynamodb get-item --table-name $TableName --item '{\"name\": {\"S\": \"John\"}}' `
| ConvertFrom-Json ).Item
\end{minted}

\subsection{Getting all items}\label{sec:getting-all-items}

\begin{minted}{bash}
aws dynamodb scan --table-name $TableName
\end{minted}

We could for example iterate over the returned items:

\begin{minted}{powershell}
$Items = (aws dynamodb scan --table-name $TableName | ConvertFrom-Json).Items
foreach ( $Item in $Items ) {
    Write-Host "$($Item.name.S ) has $($Item.points.N) points"
}
\end{minted}

\subsection{Table deletion}\label{sec:table-deletion}

\begin{minted}{bash}
aws dynamodb delete-table --table-name $TableName
\end{minted}

\section{CloudFormation}

\inputminted{yaml}{dynamodb_template.yml}

\end{document}

